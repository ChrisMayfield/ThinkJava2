%%% Section 2.9
\section{Composition}

\index{composition}

So far we have looked at the elements of a programming language -- variables, expressions, and statements -- in isolation, without talking about how to put them together.

One of the most useful features of programming languages is their ability to take small building blocks and {\bf compose} them.
For example, we know how to multiply numbers and we know how to display values.
We can combine these operations into a single statement:

\begin{code}
System.out.println(17 * 3);
\end{code}

Any arithmetic expression can be used inside a print statement.
We've already seen one example:

\begin{code}
System.out.println(hour * 60 + minute);
\end{code}

You can also put arbitrary expressions on the right side of an assignment:

\begin{code}
int percentage;
percentage = (minute * 100) / 60;
\end{code}

The left side of an assignment must be a variable name, not an expression.
That's because the left side indicates where the result will be stored, and expressions do not represent storage locations.

\begin{code}
hour = minute + 1;  // correct
minute + 1 = hour;  // compiler error
\end{code}

\index{readability}

The ability to compose operations may not seem impressive now, but we will see examples later on that allow us to write complex computations neatly and concisely.
But don't get too carried away.
Large, complex expressions can be hard to read and debug.

%Before you get too carried away with composition, keep in mind that other people will be reading your source code.
%In practice, software developers spend the vast majority of their time {\em understanding} and {\em modifying} existing code.
%Thus it's far more important to write code that is readable than to write code that is (or appears to be) optimal.
%There is much beauty in simplicity.
%In general, each line of code should be a single step of the algorithm.

