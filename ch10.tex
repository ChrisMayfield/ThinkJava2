\chapter{Classes and objects}
\label{objects}

\index{String class}
\index{type!String}

As we learned in the previous chapter, an object is a collection of data that provides a set of methods.
For example, a \java{String} is a collection of characters that provides methods like \java{charAt} and \java{substring}.

\index{object-oriented}

Java is an ``object-oriented'' language, which means that it uses objects to represent data {\em and} provide methods related to them.
This way of organizing programs is a powerful design concept, and we will introduce it a little at a time throughout the remainder of the book.

In this chapter, we introduce two new types of objects: \java{Point} and \java{Rectangle}.
We show how to write methods that take objects as parameters and produce objects as return values.
We also take a look at the source code for the Java library.


\section{Point objects}
\label{point}

\index{AWT}
\index{java.awt}
\index{Point}
\index{class!Point}

The \java{java.awt} package provides a class named \java{Point} intended to represent the coordinates of a location in a Cartesian plane.
In mathematical notation, points are often written in parentheses with a comma separating the coordinates.
For example, $(0,0)$ indicates the origin, and $(x,y)$ indicates the point $x$ units to the right and $y$ units up from the origin.

In order to use the \java{Point} class, you have to import it:

\begin{code}
import java.awt.Point;
\end{code}

\index{new}
\index{operator!new}

Then, to create a new point, you have to use the \java{new} operator:

\begin{code}
Point blank;
blank = new Point(3, 4);
\end{code}

\index{declaration}
\index{statement!declaration}

The first line declares that \java{blank} has type \java{Point}.
The second line creates the new \java{Point} with the given arguments as coordinates.

\index{reference}

The result of the \java{new} operator is a {\em reference} to the new object.
So \java{blank} contains a reference to the new \java{Point} object.
Figure~\ref{fig.reference} shows the result.

\index{state diagram}
\index{diagram!state}

\begin{figure}[!ht]
\begin{center}
\includegraphics{figs/reference.pdf}
\caption{State diagram showing a variable that refers to a \java{Point} object.}
\label{fig.reference}
\end{center}
\end{figure}

As usual, the name of the variable \java{blank} appears outside the box, and its value appears inside the box.
In this case, the value is a reference, which is represented with an arrow.
The arrow points to the new object, which contains two variables, \java{x} and \java{y}.


\section{Attributes}
\label{attribute}

\index{attribute}
\index{dot notation}

Variables that belong to an object are usually called {\bf attributes}, but you might also see them called ``fields''.
To access an attribute of an object, Java uses {\bf dot notation}.
For example:

\begin{code}
int x = blank.x;
\end{code}

The expression \java{blank.x} means ``go to the object \java{blank} refers to, and get the value of the attribute \java{x}.''
In this case, we assign that value to a local variable named \java{x}.
There is no conflict between the local variable named \java{x} and the attribute named \java{x}.
The purpose of dot notation is to identify {\em which} variable you are referring to unambiguously.

You can use dot notation as part of an expression.
For example:

\begin{code}
System.out.println(blank.x + ", " + blank.y);
int sum = blank.x * blank.x + blank.y * blank.y;
\end{code}

The first line displays \java{3, 4}; the second line calculates the value \java{25}.


\section{Objects as parameters}

\index{parameter}
\index{object!as parameter}

You can pass objects as parameters in the usual way.
For example:

\begin{code}
public static void printPoint(Point p) {
    System.out.println("(" + p.x + ", " + p.y + ")");
}
\end{code}

This method takes a point as an argument and displays its attributes in parentheses.
If you invoke \java{printPoint(blank)}, it displays \java{(3, 4)}.

But we don't really need a method like \java{printPoint}, because if you invoke \java{System.out.println(blank)} you get:

\begin{stdout}
java.awt.Point[x=3,y=4]
\end{stdout}

\java{Point} objects provide a method called \java{toString} that returns a string representation of a point.
When you call \java{println} with objects, it automatically calls \java{toString} and displays the result.
In this case, it shows the name of the type (\java{java.awt.Point}) and the names and values of the attributes.

As another example, we can rewrite the \java{distance} method from Section~\ref{distance} so that it takes two \java{Point}s as parameters instead of four \java{double}s.

\begin{code}
public static double distance(Point p1, Point p2) {
    int dx = p2.x - p1.x;
    int dy = p2.y - p1.y;
    return Math.sqrt(dx * dx + dy * dy);
}
\end{code}

Passing objects as parameters makes the source code more readable and less error-prone, because related values are bundled together.


\section{Objects as return types}

\index{Rectangle}
\index{class!Rectangle}

The \java{java.awt} package also provides a class called \java{Rectangle}.
To use it, you have to import it:

\begin{code}
import java.awt.Rectangle;
\end{code}

\java{Rectangle} objects are similar to points, but they have four attributes: \java{x}, \java{y}, \java{width}, and \java{height}.
The following example creates a \java{Rectangle} object and makes the variable \java{box} refer to it:

\begin{code}
Rectangle box = new Rectangle(0, 0, 100, 200);
\end{code}

Figure~\ref{fig.rectangle} shows the effect of this assignment.

\begin{figure}[!ht]
\begin{center}
\includegraphics{figs/rectangle.pdf}
\caption{State diagram showing a \java{Rectangle} object.}
\label{fig.rectangle}
\end{center}
\end{figure}

If you run \java{System.out.println(box)}, you get:

\begin{stdout}
java.awt.Rectangle[x=0,y=0,width=100,height=200]
\end{stdout}

Again, \java{println} uses the \java{toString} method provided by \java{Rectangle}, which knows how to display \java{Rectangle} objects.

\index{return}
\index{statement!return}

You can write methods that return objects.
For example, \java{findCenter} takes a \java{Rectangle} as an argument and returns a \java{Point} with the coordinates of the center of the rectangle:

\begin{code}
public static Point findCenter(Rectangle box) {
    int x = box.x + box.width / 2;
    int y = box.y + box.height / 2;
    return new Point(x, y);
}
\end{code}

The return type of this method is \java{Point}.
The last line creates a new \java{Point} object and returns a reference to it.


\section{Mutable objects}

\index{mutable}
\index{object!mutable}

You can change the contents of an object by making an assignment to one of its attributes.
For example, to ``move'' a rectangle without changing its size, you can modify the \java{x} and \java{y} values:

\begin{code}
Rectangle box = new Rectangle(0, 0, 100, 200);
box.x = box.x + 50;
box.y = box.y + 100;
\end{code}

The result is shown in Figure~\ref{fig.rectangle2}.

\begin{figure}[!ht]
\begin{center}
\includegraphics{figs/rectangle2.pdf}
\caption{State diagram showing updated attributes.}
\label{fig.rectangle2}
\end{center}
\end{figure}

\index{encapsulation}
\index{generalization}

We can encapsulate this code in a method and generalize it to move the rectangle by any amount:

\begin{code}
public static void moveRect(Rectangle box, int dx, int dy) {
    box.x = box.x + dx;
    box.y = box.y + dy;
}
\end{code}

The variables \java{dx} and \java{dy} indicate how far to move the rectangle in each direction.
Invoking this method has the effect of modifying the \java{Rectangle} that is passed as an argument.

\begin{code}
Rectangle box = new Rectangle(0, 0, 100, 200);
moveRect(box, 50, 100);
System.out.println(box);
\end{code}

%The code displays \java{java.awt.Rectangle[x=50,y=100,width=100,height=200]}.

Modifying objects by passing them as arguments to methods can be useful.
But it can also make debugging more difficult, because it is not always clear which method invocations modify their arguments.

Java provides a number of methods that operate on \java{Point}s and \java{Rectangle}s.
For example, \java{translate} has the same effect as \java{moveRect}, but instead of passing the rectangle as an argument, you use dot notation:

\begin{code}
box.translate(50, 100);
\end{code}

This line invokes the \java{translate} method for the object that \java{box} refers to.
As a result, the \java{box} object is updated directly.

\index{object-oriented}

This example is a good illustration of {\bf object-oriented} programming.
Rather than write methods like \java{moveRect} that modify one or more parameters, we apply methods to objects themselves using dot notation.


\section{Aliasing}
\label{aliasing}

\index{reference}

Remember that when you assign an object to a variable, you are assigning a {\em reference} to an object.
It is possible to have multiple variables that refer to the same object.
The state diagram in Figure~\ref{fig.aliasing} shows the result.

\begin{code}
Rectangle box1 = new Rectangle(0, 0, 100, 200);
Rectangle box2 = box1;
\end{code}

\begin{figure}[!ht]
\begin{center}
\includegraphics{figs/aliasing.pdf}
\caption{State diagram showing two variables that refer to the same object.}
\label{fig.aliasing}
\end{center}
\end{figure}

\index{aliasing}

Notice how \java{box1} and \java{box2} are aliases for the same object, so any changes that affect one variable also affect the other.
This example adds 50 to all four sides of the rectangle, so it moves the corner up and to the left by 50, and it increases the height and width by 100:

\begin{code}
System.out.println(box2.width);
box1.grow(50, 50);
System.out.println(box2.width);
\end{code}

The first line displays {\tt 100}, which is the width of the \java{Rectangle} referred to by \java{box2}.
The second line invokes the \java{grow} method on \java{box1}, which stretches the \java{Rectangle} horizontally and vertically.
The effect is shown in Figure~\ref{fig.aliasing2}.

\begin{figure}[!ht]
\begin{center}
\includegraphics{figs/aliasing2.pdf}
\caption{State diagram showing the effect of invoking \java{grow}.}
\label{fig.aliasing2}
\end{center}
\end{figure}

When we make a change using \java{box1}, we see the change using \java{box2}.
Thus, the value displayed by the third line is {\tt 200}, the width of the expanded rectangle.
%(As an aside, it is perfectly legal for the coordinates of a \java{Rectangle} %to be negative.)

%As you can tell from this simple example, code that involves aliasing can get confusing fast, and it can be difficult to debug.
%In general, aliasing should be avoided or used with care.


\section{The null keyword}

\index{null}

When you create an object variable, remember that you are storing a reference to an object.
In Java, the keyword \java{null} is a special value that means ``no object''.
You can declare and initialize object variables this way:

\begin{code}
Point blank = null;
\end{code}

The value \java{null} is represented in state diagrams by a small box with no arrow, as in Figure~\ref{fig.reference2}.

\begin{figure}[!ht]
\begin{center}
\includegraphics{figs/reference2.pdf}
\caption{State diagram showing a variable that contains a \java{null} reference.}
\label{fig.reference2}
\end{center}
\end{figure}

\index{exception!NullPointer}
\index{NullPointerException}
\index{run-time error}

If you try to use a \java{null} value, either by accessing an attribute or invoking a method, Java throws a \java{NullPointerException}.

\begin{code}
Point blank = null;
int x = blank.x;              // NullPointerException
blank.translate(50, 50);      // NullPointerException
\end{code}

On the other hand, it is legal to pass a null reference as an argument or receive one as a return value.
For example, \java{null} is often used to represent a special condition or indicate an error.


\section{Garbage collection}

In Section~\ref{aliasing}, we saw what happens when more than one variable refers to the same object.
What happens when {\em no} variables refer to an object?

\begin{code}
Point blank = new Point(3, 4);
blank = null;
\end{code}

The first line creates a new \java{Point} object and makes \java{blank} refer to it.
The second line changes \java{blank} so that instead of referring to the object, it refers to nothing.
In the state diagram, we remove the arrow between them, as in Figure~\ref{fig.reference3}.

\begin{figure}[!ht]
\begin{center}
\includegraphics{figs/reference3.pdf}
\caption{State diagram showing the effect of setting a variable to \java{null}.}
\label{fig.reference3}
\end{center}
\end{figure}

If there are no references to an object, there is no way to access its attributes or invoke a method on it.
From the programmer's view, it ceases to exist.
However it's still present in the computer's memory, taking up space.

\index{garbage collection}

As your program runs, the system automatically looks for stranded objects and reclaims them; then the space can be reused for new objects.
This process is called {\bf garbage collection}.

You don't have to do anything to make garbage collection happen, and in general don't have to be aware of it.
But in high-performance applications, you may notice a slight delay every now and then when Java reclaims space from discarded objects.
%You can manually run the garbage collector by invoking \java{System.gc()} method.


\section{Class diagrams}
\label{UML}

To summarize what we've learned so far, \java{Point} and \java{Rectangle} objects each have their own attributes and methods.
Attributes are an object's {\em data}, and methods are an object's {\em code}.
An object's class defines which attributes and methods it will have.

\index{UML}

In practice, it's more convenient to look at high-level pictures than to examine the source code.
{\bf Unified Modeling Language} (UML) defines a standard way to summarize the design of a class.

\begin{figure}[!ht]
\begin{center}
\includegraphics{figs/point-rect.pdf}
\caption{UML class diagrams for \java{Point} and \java{Rectangle}.}
\label{fig.umlPoint}
\end{center}
\end{figure}

\index{class diagram}
\index{diagram!class}

As shown in Figure~\ref{fig.umlPoint}, a {\bf class diagram} is divided into two sections.
The top half lists the attributes, and the bottom half lists the methods.
UML uses a language-independent format, so rather than showing \java{int x}, the diagram uses {\tt x:~int}.

In contrast to state diagrams, which visualize objects (and variables) at run-time, a class diagram visualizes the source code at compile-time.

Both \java{Point} and \java{Rectangle} have additional methods; we are only showing the ones introduced in this chapter.
See the documentation for these classes to learn more about what they can do.


\section{Java library source}

\index{library}

Throughout the book, you have used classes from the Java library including \java{System}, \java{String}, \java{Scanner}, \java{Math}, \java{Random}, and others.
You may not have realized that these classes are written in Java.
In fact, you can take a look at the source code to see how they work.

\index{src.zip}

The Java library contains thousands of files, many of which are thousands of lines of code.
That's more than one person could read and understand fully, so please don't be intimidated!

Because it's so large, the library source code is stored in a file named \java{src.zip}.
Take a few minutes to locate this file on your machine:

\begin{itemize}
\item On Linux, it's likely under: \verb"/usr/lib/jvm/openjdk-8/"
\\ (You might need to install the {\tt openjdk-8-source} package.)
\item On OS X, it's likely under: \\ \verb"/Library/Java/JavaVirtualMachines/jdk.../Contents/Home/"
\item On Windows, it's likely under: \verb"C:\Program Files\Java\jdk...\"
\end{itemize}

When you open (or unzip) the file, you will see folders that correspond to Java packages.
For example, open the {\tt java} folder and then open the {\tt awt} folder.
You should now see {\tt Point.java} and {\tt Rectangle.java}, along with the other classes in the \java{java.awt} package.

Open {\tt Point.java} in your editor and skim through the file.
It uses language features we haven't yet discussed, so you probably won't understand everything.
But you can get a sense of what professional Java software looks like by browsing through the library.

\index{documentation}
\index{HTML}
\index{Javadoc}

Notice how much of {\tt Point.java} is documentation.
Each method is thoroughly commented, including \java{@param}, \java{@return}, and other Javadoc tags.
Javadoc reads these comments and generates documentation in HTML.
You can see the results by reading the documentation for the \java{Point} class, which you can find by doing a web search for ``Java Point''.

Now take a look at \java{Rectangle}'s \java{grow} and \java{translate} methods.
There is more to them than you may have realized, but that doesn't limit your ability to use these methods in a program.

To summarize the whole chapter, objects encapsulate data and provide methods to access and modify the data directly.
Object-oriented programming makes it possible to hide messy details so that you can more easily use and understand code that other people wrote.


\section{Vocabulary}

\begin{description}

\term{attribute}
One of the named data items that make up an object.
%Each object has its own copy of the attributes for its class.

\term{dot notation}
Use of the dot operator (\java{.}) to access an object's attributes or methods.

\term{object-oriented}
A way of organizing code and data into objects, rather than independent methods.

\term{garbage collection}
The process of finding objects that have no references and reclaiming their storage space.

\term{UML}
Unified Modeling Language, a standard way to draw diagrams for software engineering.

\term{class diagram}
An illustration of the attributes and methods for a class.

\end{description}


\section{Exercises}

The code for this chapter is in the {\tt ch10} directory of {\tt ThinkJavaCode}.
See page~\pageref{code} for instructions on how to download the repository.
Before you start the exercises, we recommend that you compile and run the examples.

\begin{exercise}
The point of this exercise is to make sure you understand the mechanism for passing objects as parameters.

\begin{enumerate}

\item For the following program, draw a stack diagram showing the local variables and parameters of \java{main} and \java{riddle} just before \java{riddle} returns.
Use arrows to show which objects each variable references.

\item What is the output of the program?

\item Is the \java{blank} object mutable or immutable?
How can you tell?

\end{enumerate}

\begin{code}
public static int riddle(int x, Point p) {
    x = x + 7;
    return x + p.x + p.y;
}
\end{code}

\begin{code}
public static void main(String[] args) {
    int x = 5;
    Point blank = new Point(1, 2);

    System.out.println(riddle(x, blank));
    System.out.println(x);
    System.out.println(blank.x);
    System.out.println(blank.y);
}
\end{code}

\end{exercise}


\begin{exercise}
The point of this exercise is to make sure you understand the mechanism for returning new objects from methods.

\begin{enumerate}

\item Draw a stack diagram showing the state of the program just before \java{distance} returns.
Include all variables and parameters, and show the objects those variables refer to.

\item What is the output of this program?
(Can you tell without running it?)

\end{enumerate}

\begin{code}
public static double distance(Point p1, Point p2) {
    int dx = p2.x - p1.x;
    int dy = p2.y - p1.y;
    return Math.sqrt(dx * dx + dy * dy);
}

public static Point findCenter(Rectangle box) {
    int x = box.x + box.width / 2;
    int y = box.y + box.height / 2;
    return new Point(x, y);
}
\end{code}

\begin{code}
public static void main(String[] args) {
    Point blank = new Point(5, 8);

    Rectangle rect = new Rectangle(0, 2, 4, 4);
    Point center = findCenter(rect);

    double dist = distance(center, blank);
    System.out.println(dist);
}
\end{code}

\end{exercise}


\begin{exercise}
This exercise is about aliasing.
Recall that aliases are two variables that refer to the same object.

\begin{enumerate}

\item Draw a diagram that shows the state of the program just before the end of \java{main}.
Include all local variables and the objects they refer to.

\item What is the output of the program?

\item At the end of \java{main}, are \java{p1} and \java{p2} aliased?
Why or why not?

\end{enumerate}

\begin{code}
public static void printPoint(Point p) {
    System.out.println("(" + p.x + ", " + p.y + ")");
}

public static Point findCenter(Rectangle box) {
    int x = box.x + box.width / 2;
    int y = box.y + box.height / 2;
    return new Point(x, y);
}
\end{code}

\begin{code}
public static void main(String[] args) {
    Rectangle box1 = new Rectangle(2, 4, 7, 9);
    Point p1 = findCenter(box1);
    printPoint(p1);

    box1.grow(1, 1);
    Point p2 = findCenter(box1);
    printPoint(p2);
}
\end{code}

\end{exercise}


\begin{exercise}
\label{ex.biginteger}

\index{factorial}

You might be sick of the factorial method by now, but we're going to do one more version.

\begin{enumerate}

\item Create a new program called {\tt Big.java} and write (or reuse) an iterative version of \java{factorial}.

\item Display a table of the integers from 0 to 30 along with their factorials.
At some point around 15, you will probably see that the answers are not right anymore.
Why not?

\index{BigInteger}

\item \java{BigInteger} is a Java class that can represent arbitrarily big integers.
There is no upper bound except the limitations of memory size and processing speed.
Take a minute to read the documentation, which you can find by doing a web search for ``Java BigInteger''.

\item To use BigIntegers, you have to import \java{java.math.BigInteger} at the beginning of your program.

\item There are several ways to create a BigInteger, but the simplest uses \java{valueOf}.
The following code converts an integer to a BigInteger:

\begin{code}
int x = 17;
BigInteger big = BigInteger.valueOf(x);
\end{code}

%Type in this code and try it out.
%Try displaying a BigInteger.

\item Since BigIntegers are not primitive types, the usual math operators don't work.
Instead, we have to use methods like \java{add}.
To add two BigIntegers, invoke \java{add} on one and pass the other as an argument.

\begin{code}
BigInteger small = BigInteger.valueOf(17);
BigInteger big = BigInteger.valueOf(1700000000);
BigInteger total = small.add(big);
\end{code}

Try out some of the other methods, like \java{multiply} and \java{pow}.

\item Convert \java{factorial} so that it performs its calculation using BigIntegers and returns a BigInteger as a result.
You can leave the parameter alone; it will still be an integer.

\item Try displaying the table again with your modified factorial method.
Is it correct up to 30?
How high can you make it go?

\item Are BigInteger objects mutable or immutable?
How can you tell?

\end{enumerate}
\end{exercise}


\begin{exercise}
Many encryption algorithms depend on the ability to raise large integers to a power.
Here is a method that implements an efficient algorithm for integer exponentiation:

\begin{code}
public static int pow(int x, int n) {
    if (n == 0) return 1;

    // find x to the n/2 recursively
    int t = pow(x, n / 2);

    // if n is even, the result is t squared
    // if n is odd, the result is t squared times x
    if (n % 2 == 0) {
        return t * t;
    } else {
        return t * t * x;
    }
}
\end{code}

\index{BigInteger}

The problem with this method is that it only works if the result is small enough to be represented by an \java{int}.
Rewrite it so that the result is a \java{BigInteger}.
The parameters should still be integers, though.

You should use the \java{BigInteger} methods \java{add} and \java{multiply}.
But don't use \java{BigInteger.pow}; that would spoil the fun.
\end{exercise}
